\Kapitel{Brot und Brötchen}{pics/brot}
\section{Sauerteig}
\begin{zutaten}
260g&Mehl\\
260ml&Wasser\\
\end{zutaten}
\begin{steps}
 \step Tag 1: 30g Mehl und 30ml Wasser in einem verschließbaren Gefäß vermengen und offen stehen lassen.
 \step Tag 2-4: 30g Mehl und 30ml Wasser zum Ansatz hinzufügen und Gefäß offen stehen lassen.
 \step Tag 5: 30g vom Ansatz herauswiegen und mit 70g Mehl und 70ml Wasser vermengen und offen stehen lassen. Den Rest entsorgen.
 \step Tag 6: 70g Mehl und 70ml Wasser zum Ansatz hinzufügen und Gefäß offen stehen lassen.
 \step Tag 7: Erstes Brot kann gemacht werden. Bei den ersten Versuchen etwas Trockenhefe dazugeben, da der Sauerteig noch nicht genug Kraft entwickelt hat.
\end{steps}
\section{Sauerteigbrot}
\begin{zutaten}
140g&Mehl\\
140ml&Wasser\\
30g&Sauerteigansatz\\
500g&Mehl\\
450ml&Wasser\\
10g&Salz\\
4EL&Olivenöl\\
\end{zutaten}
\begin{steps}
 \step Am Vortag Sauerteigansatz mit 140g Mehl und 140ml Wasser vermengen und verschlossen stehen lassen.
 \step Vom Sauerteigansatz 30g herauswiegen und für das nächste Mal verschlossen in den Kühlschrank stellen.
 \step Den Sauerteigansatz im Wasser auflösen.
 \step Mit Mehl, Salz und Olivenöl vermengen und für 10-15 Minuten kneten.
 \step Gärkörbchen oder Kastenform ausmehlen und den Teig über Nacht (min. 8 Stunden) zugedeckt an einem warmen Ort gehen lassen.
 \step Ofen auf 250°C Umluft mit einer Schale Wasser vorheizen.
 \step Brot aus dem Gärkörbchen stürzen und für 50 Minuten backen. Nach 10 Minuten Temperatur auf 220°C reduzieren.
\end{steps}
\Kapitel{Kuchen und Torten}{pics/torte}
\section{Schwarzwälder Kirschtorte}
\begin{zutaten}
5&Eier\\
1 Prise&Salz\\
200g&Zucker\\
2 Pkg&Vanillezucker\\
250g&Mehl\\
1 Pkg&Backpulver\\
3 EL&Kakaopulver\\
1 EL&Rum\\
800g&Kirschen\\
400g&Gelierzucker 2:1\\
500ml&Schlagobers\\
2 Pkg&Sahnesteif\\
1 Schuss&Zitronensaft\\
200g&Schokoraspeln\\
\end{zutaten}
\begin{steps}
 \step Kirschen entkernen und mit dem Gelierzucker zum Kochen bringen, anschließend köcheln lassen.
 \step Ofen auf 190°C (Umluft 175°C) vorheizen. Eier, Mehl, Zucker, Vanillezucker, Backpulver, Kakao und Rum in einer Rührschüssel glatt rühren.
 \step Teig in eine Springform geben und für 25-30 Minuten backen.
 \step Den fertigen Boden in 3 Teile schneiden.
 \step Boden und Kirschmasse kühlen lassen.
 \step Schlagobers mit Sahnesteif steif schlagen und Zitronensaft zugeben.
 \step Den ersten Boden auf eine Platte legen und den Springformrand oder einen Toretenring darum herum stellen. Darauf die Kirschmasse gleichmäßig verteilen.
 \step Den zweiten Boden auflegen und leicht andrücken. Ein Drittel des Schlagobers gleichmäßig darauf streichen.
 \step Den dritten Boden auflegen und leicht andrücken. Den restlichen Schlagobers gleichmäßig auf und um die Torte streichen. Mit Schokoraspeln garnieren.
 \step 16 Tupfen Schlagobers gleichmäßig am Rand der oberfläche verteilen und darauf entkernte Kirschen platzieren. Torte vor dem servieren 3-4 Stunden kaltstellen.
\end{steps}
